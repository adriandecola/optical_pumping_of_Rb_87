\chapter*{Introduction}
% The asterisk prevents this file from being labelled
% as a 'chapter.'

Occasionally referred to as the fifth state of matter, Bose-Einstein Condensation pushes the extremes of our physical universe. Super-cooled to within a millionth of a degree of absolute zero, these particles harmoniously lose their individuality, joining together to form a 'quantum super particle'. This extraordinary state of matter, Bose-Einstein Condensation, pushed the frontier of physics and remains a great example of generational achievement and scientific teamwork. Today, Bose-Einstein Condensation remains a heavily interesting field researched in areas such as supersolids, quantum computers, and precision measurement. 


Against the backdrop of World War I, physicist and mathematician Satyendra Bose sent a paper to Albert Einstein about the statistical relationship between light and temperature, where he treated photons as indistinguishable particles. Einstein, impressed with his work, translated and submitted the paper on his behalf to a physics journal where it was later published. Einstein extended Bose's statistics to indistinguishable atoms, resulting in Bose-Einstein Statistics. These indistinguishable atoms are part of a family of particles called bosons. With this framework, Einstein formulated a state of matter where a cloud of atoms could be cooled enough until they all condense to the same, lowest, quantum state. This new state of matter is what we refer to as Bose-Einstein Condensation. Einstein published his findings in 1925. At the time, he described the state of matter, saying "this appears to be as good as it is impossible." Such a state of matter at the time appeared to many, as nonsensical, with atoms overlapping and all sharing the same properties. 


It wasn't until 70 years later, in 1995, with new advances in laser cooling, magnetic trapping, evaporative cooling, vacuum technology, and theoretical understanding that a Bose-Einstein Condensate was finally realized, using Rubidium atoms much like the Ultra-Cold Atom Lab at Bates. Eric Cornell, Carl Wieman, and Wolfgang Ketterle received the 2001 Nobel Prize in Physics for being the first labs to produce Bose-Einstein Condensates.


In the past 38 year, Bose-Einstein Condensate technology has greatly improved. There are now hundreds of labs around the world with functioning Bose-Einstein Condensate machines. Scientists have made Bose-Einstein Condensates out of many other bosons including Alkali atoms such as Rubidium, Sodium, Cesium, Lithium, and others; noble gasses like Helium-4; and even photons. Bose-Einstein Condensates are especially interesting as they demonstrate quantum behavior on a macroscopic scale, can display unique properties such as super fluidity and coherence, and because of their tunability, which is useful for simulation. Due to these properties, Bose-Einstein Condensates remain among the leading areas of research; especially in condensed matter physics, quantum simulation and information, and precision measurements. 


Over the past years, Nathan Lundblad, Bates students, and post-doctorates have worked to create an atom chip based Bose-Einstein Condensate machine. This atom chip based machine will complement the currently functional Bose-Einstein Condensate machine in the lab. The motivation behind another machine is to build a smaller, lighter, easily controlled Bose-Einstein Condensate machine, much like the one in the Cold Atom Lab on the International Space Station. This new machine will be useful for simulating experiments we hope to perform on the International Space Station. One example of this is the realization of a 'bubble-like' Bose-Einstein Condensate. Creating a magnetic field gradient for a Bose-Einstein Condensate of this shape is not possible in the presence of Earth's gravitational field. Realizing one on the International Space Station has not yet been possible and an atom chip based Bose-Einstein Condensate machine could help in the diagnostics of this quest. An atom chip based Bose-Einstein Condensate machine has the potential to open many other avenues of research with its precise control systems, sleekness, and its utility to simulate experiments on the Cold Atom Lab on the International Space Station. 


This thesis builds upon the work done by previous students. To create a deep, intuitive understanding of the project, this thesis first provides a theoretical framework for which to under Bose-Einstein Condensation and the process for its experimental realization. This thesis works to advances the creation of an optical pumping system to purify the Rubidium-87 atoms into a single substate, $m_F = 2$. The second part of my thesis works to computationally simulate the optical pumping process. This simulation gives a deeper understanding to the process of purifying the atoms. It also serves to provide numerical values to optimizing the laser's time scattering atoms, its necessary frequency detuning, and its intensity. Laser characteristic inputs can be changed and recalculated easily for future lab use. The third part of my thesis works towards creating the optical pumping process in the Ultra-Cold Atom Lab at Bates College. This building process uses optical elements including optical fibers, linear polarizers, beam splitters, quarter-wave plates, mirrors, shutters, many mounts, and other elements. Using these elements, we establish a positive circularly polarized light from a coupled optical fiber mounted closely along the axis of the magnetic coils of the ColdQuanta quadruple coil assembly. 
 
 
 From this point, the laser can be correctly frequency detuned by the amount predicted in my optical pumping simulation. Using the ColdQuanta quadruple coil assembly, all the atoms can be effectively trapped by a magnetic field for further cooling en route to forming a Bose-Einstein Condensate. Lastly, my thesis outlines future steps for students to create a functional Bose-Einstein Condensate machine.


 